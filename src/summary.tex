\section{回顾与总结}
\label{sec:sum}
\noindent \textbf{回顾:}\\
\indent 我们将MLT中路径贡献函数记作$f(x)$,其中$x$是primary space中的一点。通常在MLT中,我们一般将$f(x)$作为目标分布函数,实际上,我们也可以改变这一目标分布(新分布记作$g(x)$),同时将样本权重调整为($\frac{f(x)}{g(x)}$),依然可以得到unbiased结果。对于包含sds路径的场景,即使是primary space下的$f(x)$,其有效区域依然是嵌入在全空间的低维流形。因此,我尝试将$g(x)$取为更加平坦的函数(如将$g(x)$取为场景中Specular物体的roughness放大后对应的路径贡献)。\textcolor{red}{$g(x)$的跳转接受率更高,但同时也会由于概率分布和真实贡献不同,引入额外的方差。(参见 \boldnameref{sec:cost})}\\

\noindent \textbf{实验总结:}\\
\indent \textcolor{red}{在本阶段的实验中,我们发现随着场景中specular物体的增多,$g(x)$逐渐体现出更大的优势(MSE比$f(x)$的结果更低)。直观上的解释是:Specular物体的增多会导致完全不连通的sds pattern增多,遍历这些pattern只能通过large step跳转样本;而$g(x)$的large step接受率更高(Table \ref{tab:accept}),使得这些pattern之间的整体亮度比例可以更快地收敛。(实验验证 参见 \boldnameref{sec:benefit})}

%另一方面,目标分布与实际的差异会引入额外的方差(可利用柯西积分不等式证明)。直观的解释是,在MH采样过程中部分样本stuck多次且$\frac{f(x)}{g(x)}$较大时,会放大误差,从而产生noise点。样本权重$\frac{f(x)}{g(x)}$较小时,也会降低稀释noise的能力。(参见 \boldnameref{sec:cost})
