\begin{figure}
\begin{minipage}{\textwidth}
\centering  
\renewcommand{\arraystretch}{0.35}
\addtolength{\tabcolsep}{-5.0pt}
 \begin{tabular}{ ccccc }
\begin{overpic}[width=\gridMseFigWidth]{\GridMSE{Pyramid}{1}{h2mc}}\end{overpic}
& \begin{overpic}[width=\gridMseFigWidth]{\GridMSE{Pyramid}{4}{h2mc}}\end{overpic}
& \begin{overpic}[width=\gridMseFigWidth]{\GridMSE{Pyramid}{9}{h2mc}}\end{overpic}
& \begin{overpic}[width=\gridMseFigWidth]{\GridMSE{Pyramid}{25}{h2mc}}\end{overpic}
& \begin{overpic}[width=\gridMseFigWidth]{\GridMSE{Pyramid}{100}{h2mc}}\end{overpic}
\\
\begin{overpic}[width=\gridMseFigWidth]{\GridMSE{Pyramid}{1}{mala}}\end{overpic}
& \begin{overpic}[width=\gridMseFigWidth]{\GridMSE{Pyramid}{4}{mala}}\end{overpic}
& \begin{overpic}[width=\gridMseFigWidth]{\GridMSE{Pyramid}{9}{mala}}\end{overpic}
& \begin{overpic}[width=\gridMseFigWidth]{\GridMSE{Pyramid}{25}{mala}}\end{overpic}
& \begin{overpic}[width=\gridMseFigWidth]{\GridMSE{Pyramid}{100}{mala}}\end{overpic}
\\
\end{tabular}
\end{minipage}
\caption{Grid[M]PyramidMSE:Diffuse平面上摆放了M个透明金字塔,从左到右依次是M=1, 4, 9, 25, 100。第一行是h2mc的结果,第二行是mala的结果。图例中C128表示chain数量为128, P0.2表示large step的概率为0.2, roughScale表示$g(x)$对$\alpha$的放大程度,1.0表示$g(x)=f(x)$。从图中可以看到,随着M数量的增加,\textcolor{orange}{$g_{1.25}(x)$}和\textcolor{green}{$g_{1.5}(x)$}会优于\textcolor{blue}{$f(x)$}}
\label{fig:GridPyramidMSE} 
\end{figure}

\begin{figure}
\begin{minipage}{\textwidth}
\centering  
\addtolength{\tabcolsep}{-5.0pt}
 \begin{tabular}{ ccccc }
\begin{overpic}[width=\gridMseFigWidth]{\GridMSE{Sphere}{1}{h2mc}}\end{overpic}
& \begin{overpic}[width=\gridMseFigWidth]{\GridMSE{Sphere}{4}{h2mc}}\end{overpic}
& \begin{overpic}[width=\gridMseFigWidth]{\GridMSE{Sphere}{9}{h2mc}}\end{overpic}
& \begin{overpic}[width=\gridMseFigWidth]{\GridMSE{Sphere}{25}{h2mc}}\end{overpic}
& \begin{overpic}[width=\gridMseFigWidth]{\GridMSE{Sphere}{100}{h2mc}}\end{overpic}
\\
\end{tabular}
\end{minipage}
\caption{Grid[M]SphereMSE:Diffuse平面上摆放了M个透明玻璃球,从左到右依次是M=1, 4, 9, 25, 100。Small Step为H2MC。}
\label{fig:GridSphereMSE} 
\end{figure}

\begin{figure}
\begin{minipage}{\textwidth}
\centering  
\addtolength{\tabcolsep}{-5.0pt}
 \begin{tabular}{ ccccc }
\begin{overpic}[width=\gridMseFigWidth]{\GridMSE{Teapot}{1}{mala}}\end{overpic}
& \begin{overpic}[width=\gridMseFigWidth]{\GridMSE{Teapot}{4}{mala}}\end{overpic}
& \begin{overpic}[width=\gridMseFigWidth]{\GridMSE{Teapot}{9}{mala}}\end{overpic}
& \begin{overpic}[width=\gridMseFigWidth]{\GridMSE{Teapot}{25}{mala}}\end{overpic}
& \begin{overpic}[width=\gridMseFigWidth]{\GridMSE{Teapot}{100}{mala}}\end{overpic}
\\
% \begin{overpic}[width=\gridMseFigWidth]{\GridMSE{Teapot}{1}{h2mc}}\end{overpic}
% & \begin{overpic}[width=\gridMseFigWidth]{\GridMSE{Teapot}{4}{h2mc}}\end{overpic}
% & \begin{overpic}[width=\gridMseFigWidth]{\GridMSE{Teapot}{9}{h2mc}}\end{overpic}
% & \begin{overpic}[width=\gridMseFigWidth]{\GridMSE{Teapot}{25}{h2mc}}\end{overpic}
% & \begin{overpic}[width=\gridMseFigWidth]{\GridMSE{Teapot}{100}{h2mc}}\end{overpic}
% \\
\end{tabular}
\end{minipage}
\caption{Grid[M]TeapotMSE 在Diffuse平面上摆放了M个透明茶壶,从左到右依次是M=1, 4, 9, 25, 100。Small Step为MALA。}
\label{fig:GridTeapotMSE} 
\end{figure}