\section{算法配置与实验场景}
\label{sec:desc}
\subsection{算法配置说明}
本次实验中透明材质为rough dielectric, 其NDF采用Beckman模型,粗糙度由$\alpha$描述;反射材质为phong模型(mala源码支持的模型),粗糙度由exponent描述。\textcolor{red}{本报告中默认$g_k(x)$表示将$\alpha$放大$k$倍,exponent缩小$k$倍后得到的路径贡献。本报告中出现的$g_{1.25}$和$g_{1.5}$对应于$k=1.25$和$k=1.5$。}由于MALA的源码中,direct path是直接渲染的,MLT只需要负责indirect path (bounce次数大于等于2)。因此,\textcolor{red}{本报告中所有图片都是render indirect path的图片,并将其normalize到相同平均亮度以排除不必要的干扰。包含direct和indirect的reference附于本报告最后,以供参考。}

\subsection{实验场景}
如无特殊标注,光源为一个很小的面光源,位于场景斜上方,本报告涉及到的场景如下:

\begin{enumerate}[(1)]
    \item \textbf{Grid[M]Pyramid}: Diffuse的地板上放置M个透明的金字塔,位置上排成网格,M=1,4,9,25,100。
    \item \textbf{Grid[M]Sphere}: Diffuse的地板上放置M个透明的圆球,位置上排成网格,M=1,4,9,25,100。
    \item \textbf{Grid[M]Teapot}: Diffuse的地板上放置M个透明茶壶(使用的是Veachdoor场景中的mesh),位置上排成网格,M=1,4,9,25,100。
    % \item \textbf{Grid[M]TeapotRefl}: Diffuse的地板上放置M个phong材质的茶壶(使用的是Veachdoor场景中的mesh),位置上排成网格,M=1,4,9,25,100。
    \item \textbf{PlaneSphere05}: Diffuse的地板上放置了半个表面有所起伏的透明球(05只是建模时试了不同起伏程度的编号,不用管)。
    \item \textbf{Slab}: SMS文中的场景。
    % \item \textbf{PlaneSlab}: 将Slab的曲面横置于Diffuse的地板上。
    \item \textbf{Veachdoor}: 同MALA源码中的场景,门缝外有一个较大的面光源。
    \item \textbf{Torus}: 同MALA源码中的场景,光源为环境光sunsky。
\end{enumerate}






